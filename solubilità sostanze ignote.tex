\documentclass{article}
\usepackage[utf8]{inputenc}
\usepackage[italian]{babel}
\usepackage{multicol}
\usepackage{amssymb}
\usepackage{chemfig}
\title{Prove di solubilità su sostanze incognite}
\author{Cristiano Colpo cl. 4AC}
\date{November 2014}

\begin{document}

\maketitle

\section{Obiettivo}
L'obiettivo di questa esperienza è quello di determinare la classe di solubilità di alcune sostanze incognite e eseguire il saggio di tollens per individuare un aldeide o un chetone. 

\section{Materiale}
\begin{multicols}{2}
\renewcommand{\labelitemi}{$\square$}
\begin{itemize}
    \item porta provette
    \item pipetta graduata
    \item provette con tappo
    \item spruzzetta
    \item cartina tornasole
    \item bacchetta di vetro
    \item becher
    \item pipetta conta gocce
    \item matracci
\end{itemize}
\end{multicols}

\section{Sostanze}
\begin{multicols}{2}
\renewcommand{\labelitemi}{$\square$}
\begin{itemize}
    \item 4 sostanze incognite (A, B, C, D)
    \item \chemfig{H_2 O} distillata
    \item soluzione di etere
    \item soluzione di NaOH al 5\%
    \item soluzione di HCl al 5\%
    \item soluzione di \chemfig{NaHCO_3} al 5\%
    \item soluzione di \chemfig{NH_3}
    \item soluzione di \chemfig{NH_4 OH}
    \item soluzione di \chemfig{AgNO_3}
\end{itemize}
\end{multicols}

\section{Sicurezza}
Utilizzare guanti e occhiali protettivi con il reattivo di Tollens.

\section{Procedimento}
\begin{enumerate}
\item Eseguire le prove di solubilità seguendo lo schema sottostante
\vspace{10mm}
\vfill
\item Per riconoscere aldeidi o chetoni si fa uso del saggio di Tollens: si prepara il reattivo a fresco perché può essere esplosivo, secondo la reazione:

\schemestart
    \chemfig{2Ag^{+}NO^{-}_3}
    \+
    \chemfig{2NaOH}
    \arrow(.mid east--.mid west){->}
    \chemfig{Ag_2 O_{(S)}}
    \+
    \chemfig{H_2 O}
    \+
    \chemfig{Na^{+}NO^{-}_3}
\schemestop

\schemestart
    \chemfig{Ag_2 O_{(S)}}
    \+
    \chemfig{4NH_3}
    \+
    \chemfig{H_2 O}
    \arrow(.mid east--.mid west){->}
    \chemfig{2Ag{(}NH_3{)}^{+}_2}
    \+
    \chemfig{OH^{-}}
\schemestop

Il prodotto che otteniamo \chemfig{2Ag{(}NH_3{)}^{+}_2} è il nitrato d'argento ammoniacale chiamato anche \texttit{reattivo di Tollens}.
\par
Si prende quindi una provetta con circa 2mL di \chemfig{2Ag^{+}NO^{-}_3} e 0,5mL di \chemfig{NaOH}, si lascia poi a riposo fino a quando non si forma un precipitato. Successivamente si aggiunge idrossido di ammonio \chemfig{NH_{4}OH} fino a solubilizzare completamente il precipitato (meglio utilizzare la pipetta).
A questo punto il reattivo è pronto, si aggiunge in provetta qualche goccia della sostanza incognita e si osserva se si forma o no uno specchio d'argento.
\par
Se si osserva lo specchio d'argento la sostanza è quindi un aldeide perchè a differenza dei chetoni, lo ione argento complessato con l'ammoniaca può essere ridotto solo dalle aldeidi ad argento metallico e non dai chetoni che non reagiscono.
\end{enumerate}

\section{Dati e risultati}
\begin{table}[h]
\begin{tabular}{|c|c|c|c|c|c|c|c|}
\hline
  & \chemfig{H_2 O} & Etere & NaOH & HCl & \chemfig{NaHCO_3} & \chemfig{H_{2}SO_4} & CLASSE \\\hline
A &        si            &  si     &      &     &     &      &  S_a      \\\hline
B &         si           &  insol     &      &     &     &      &  S      \\\hline
C &           insol         &       &   insol   &  insol   &     &   ?   &       ?\\\hline
  & \multicolumn{6}{c|}{COSA SI VERIFICA}              & SOSTANZA \\\hline
D & \multicolumn{6}{c|}{formazione specchio d'argento} & aldeide \\\hline
\end{tabular}
\end{table}

\section{Conclusioni}
Abbiamo determinato le tre classi di solubilità ( A $\rightarrow$ S_A , B  $\rightarrow$ S ) ad eccezione della sostanza C perché non ci era permesso utilizzare l'acido solforico 96\% e abbiamo identificato l'ultima sostanza D come un aldeide.
\end{document}
