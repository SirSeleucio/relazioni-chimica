\documentclass{article}
\usepackage[utf8]{inputenc}
\usepackage[italian]{babel}
\usepackage{multicol}
\usepackage{amssymb}
\usepackage{chemfig}
\usepackage{graphicx}
\usepackage{textcomp}
\title{Titolazione dell'acido lattico contenuto nel latte}
\author{Cristiano Colpo cl. 4AC}
\date{November 2014}

\begin{document}

\maketitle

\section{Obiettivo}
L'obiettivo di questa esperienza è quello di ricavare il volume di titolante con cui si neutralizza l'acido lattico e ricavare la relativa acidità equivalente espressa in \textdegree SH. 

\section{Materiale}
\begin{multicols}{2}
\renewcommand{\labelitemi}{$\square$}
\begin{itemize}
    \item Buretta di Mohr
    \item Becher
    \item Pipetta graduata
    \item Cilindro graduato
    \item Ancoretta
    \item Siringa
    \item Agitatore magnetico
    \item Matracci
    \item Pipetta graduata
    \item Cilindro graduato
\end{itemize}
\end{multicols}

\section{Sostanze}
\begin{multicols}{2}
\renewcommand{\labelitemi}{$\square$}
\begin{itemize}
    \item Acqua distillata
    \item NaOH 0,1M
    \item Latte (con presenza al suo interno di acido lattico)
    \item Fenolftaleina \includegraphics[scale=0.5]{fenolftaleina}
\end{itemize}
\end{multicols}

\section{Sicurezza}
NaOH : \includegraphics[scale=0.6]{GHS05}(GHS05)Frasi H	314 - 290; Consigli P	280 - 301+330+331 - 305+351+338 - 309+310.

Fenolftaleina: \includegraphics[scale=0.6]{GHS08}(GHS05)Frasi H	341 - 350 - 361; Consigli P	201 - 281 - 308+313.

\section{Procedimento}
\begin{enumerate}
    \item Prelevare 50mL di latte con la pipetta graduata e versarli nel bacher
    \item Aggiungere 8 gocce di fenolftaleina
    \item Predisporre una buretta con la soluzione titolante di NaOH
    \item Titolare il latte presente nel bacher procedendo goccia a goccia fino a che non assume una colorazione rosata. 
\end{enumerate}

\section{Dati e risultati}
Il viraggio del indicatore è avvenuto nella prima prova con 9,8mL di titolante mentre nella seconda con 9,6mL. L'acidità del latte viene ufficialmente indicata come \textit{grado di acidità} (\textdegree SH), che corrisponde ai mL di NaOH 0,25M necessari per portare 100mL di latte a pH 8,3 (il pH di viraggio della fenolftaleina). Noi abbiamo utilizzato NaOH 0,1M ma a noi serve il volume di NaOH 0,25M pertanto utilizziamo la formula:
\begin{equation}
V_{NaOH 0,25M} = \frac{(M \times V)_{NaOH a.titolo.noto}}{0,25M} = \frac{0,1\frac{mol}{L}\times (0,0097\times 2)L}{0,25\frac{mol}{L}} = 0,00776L
\end{equation}

Da questa formula abbiamo ricavato il volume di NaOH 0,25M necessario per portare 100mL di latte a pH 8,3. Se 1mL di NaOH 0,25M è uguale a 1\textdegree SH allora 7,76mL di NaOH 0,25M corrispondono a 7,76 \textdegree SH.

Calcoliamo la percentuale di acido lattico presente nel latte con l'equazione.
\begin{equation}
acido lattico\% = °SH \times 0,0225 = 7,76 \times 0,0225 = 17,4\%
\end{equation}
Un buon latte fresco contiene solo una modesta percentuale di acido lattico (0,15-0,16\%), il nostro con il valore di 17,4\% si avvicina a questi valori.
\section{Conclusioni}
Il latte fresco ha un'acidità compresa tra 6,5 e 7,5 \textdegree SH, mentre il latte utilizzato per questa esperienza ha un'acidità di 7,76 \textdegree SH e quindi potrebbe non essere considerato \textit{fresco}. 
\end{document}
